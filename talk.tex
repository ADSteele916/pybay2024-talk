% !TeX program = xelatex

\documentclass[8pt, aspectratio=169]{beamer}

\usepackage{amsmath}
\usepackage{hyperref}
\usepackage{minted}
\usepackage{multicol}
\usepackage{unicode-math}

\setlength{\columnsep}{0pt}

\usemintedstyle{monokai}
\setminted{autogobble}
\setminted{baselinestretch=1.0}
\newminted{rust}{}
\newmintinline{rust}{}

\title{Architecting Native Extension Modules for Python}
\author{Alex Steele}
\date{September 21, 2024}
\institute{PyBay 2024}

\usetheme[progressbar=frametitle]{metropolis}

\setmonofont{CMU Typewriter Text}

\definecolor{VividCerulean}{HTML}{009de0}  
\definecolor{MaastrichtBlue}{HTML}{011936}
\definecolor{BlueSapphire}{HTML}{19647e}
\definecolor{DarkSlate}{HTML}{204B57}
\definecolor{MidnightBlue}{HTML}{1b3b6f}
\definecolor{SpaceCadet}{HTML}{21295c}
\definecolor{DutchWhite}{HTML}{a4c2a5}
\definecolor{Box1}{HTML}{432e36}
\definecolor{Box2}{HTML}{260c1a}
\definecolor{Alert1}{HTML}{e55829}
\definecolor{Alert2}{HTML}{bc4622 }
\definecolor{Example1}{HTML}{2f9c95}
\definecolor{Example2}{HTML}{3f3f3f}%{21706b}
\definecolor{CodeBG}{HTML}{131515}
\definecolor{CodeFG}{HTML}{2b2c28}
\definecolor{BackgroundColor}{HTML}{191919}

\setbeamercolor{frametitle}{bg=BackgroundColor, fg=white}
\setbeamercolor{frame}{bg=BackgroundColor, fg=white}
\setbeamercolor{progress bar}{fg=VividCerulean, bg=SpaceCadet}
\setbeamercolor{background canvas}{bg=BackgroundColor, fg=white}
\setbeamercolor{palette primary}{fg=white, bg=SpaceCadet}
\setbeamercolor{normal text}{fg=white, bg=SpaceCadet} 
\setbeamercolor{block title}{fg=white, bg=Alert1}
\setbeamercolor{block body}{fg=white, bg=Alert2}
\setbeamercolor{block title alerted}{fg=white, bg=Alert1}
\setbeamercolor{block body alerted}{fg=white, bg=Alert2}
\setbeamercolor{block title example}{fg=white, bg=Example1}
\setbeamercolor{block body example}{fg=white, bg=Example2}
\setbeamercolor{alerted text}{fg=VividCerulean}
\setbeamercolor{itemize item}{fg=white}

\metroset{block=fill}

% Extremely cursed hack to get the modulo operator in a \mintinline.
% https://tex.stackexchange.com/questions/331695
\ExplSyntaxOn
\NewDocumentCommand{\mintedstring}{mv}
 {
  \tl_clear_new:c { l_minted_string_#1_tl }
  \tl_set:cn { l_minted_string_#1_tl } { #2 }
 }
\NewDocumentCommand{\mintinlinestring}{mm}
 {
  \minted_inline_string:nv { #1 } { l_minted_string_#2_tl }
 }
\cs_new_protected:Nn \minted_inline_string:nn
 {
  \mintinline{#1}{#2}
 }
\cs_generate_variant:Nn \minted_inline_string:nn { nv }
\ExplSyntaxOff

\setbeamertemplate{footline}
{
  \leavevmode%
  \hbox{%
  \begin{beamercolorbox}[wd=.3\paperwidth,ht=2.25ex,dp=1ex,center]{frametitle}%
    \usebeamerfont{author in head/foot}\insertshortauthor \ (\insertshortinstitute)
  \end{beamercolorbox}%
  \begin{beamercolorbox}[wd=.47\paperwidth,ht=2.25ex,dp=1ex,center]{frametitle}%
    \usebeamerfont{title in head/foot}\insertshorttitle\hspace*{3em}
  \end{beamercolorbox}}%
  \begin{beamercolorbox}[wd=.23\paperwidth,ht=2.25ex,dp=1ex,center]{frametitle}%    
    \insertframenumber{} \hspace*{1ex}
  \end{beamercolorbox}%
  \vskip0pt%
}

\begin{document}

\maketitle

\begin{frame}{Background}
\begin{definition}[Native Extension Module]
A Python module that is written in a language that compiles to native code and exposes new ``built-in'' objects to the interpreter.
\end{definition}
\begin{itemize}
\item NumPy, SciPy, pandas, PyTorch, Pydantic, Polars, and Pillow all use native extensions for some of their functionality.
\item The classic way to make an extension module is the Python API for C, but several high-level frameworks now exist for creating extension modules.
\begin{itemize}
\item Boost.Python and pybind11 for C++.
\item PyO3 for Rust.
\end{itemize}
\end{itemize}
\end{frame}

\begin{frame}{Decisions}
\begin{itemize}
\item Native extensions are great.
\begin{itemize}
\item Avoid re-inventing the wheel.
\item Drastically improve performance.
\item Interact with hardware and low-level systems.
\end{itemize}
\item But:
\begin{itemize}
\item System complexity increases.
\item Portability is often a challenge.
\item Tooling is annoying.
\end{itemize}
\item Ensure that this is a trade-off you and your team are willing to make.
\begin{itemize}
\item ``Why do we need Python \emph{and} C++?''
\end{itemize}
\end{itemize}
\end{frame}

\begin{frame}{Performance}
\begin{itemize}
\item Calls and type conversions across the foreign function interface (FFI) are expensive. Make them as infrequent as possible.
\begin{itemize}
\item Don't feel obligated to copy your C/C++/Rust API.
\item Use context managers.
\item Batch operations.
\end{itemize}
\item The global interpreter lock (GIL) is easier to work around in native code. Parallelize if it makes sense.
\item As always, write benchmarks and profile them to find hot points in your code.
\begin{itemize}
\item \rustinline|py-spy| is a sampling profiler that supports native extensions.
\end{itemize}
\end{itemize}
\end{frame}

\begin{frame}{Architecture}
\begin{itemize}
\item Dealing with Python FFI should be a separate concern from the business logic of your library.
\begin{itemize}
\item Making your core native library and your Python bindings into separate packages can help enforce this.
\end{itemize}
\item Your extension module API should be written with Python users in mind.
\begin{itemize}
\item Much of the power of Python lies in its simplicity and accessibility. Don't make users write C++ in Python.
\end{itemize}
\end{itemize}
\end{frame}

\begin{frame}{Key Takeaways}
\begin{itemize}
\item Only use a native extension if you're getting value from both Python and a natively-compiled language.
\item FFI calls are expensive, minimize if possible.
\item You're still writing a Python module. Make it accessible to Python users.
\end{itemize}
\end{frame}

{
\setbeamertemplate{footline}{}
\begin{frame}[standout]
Thank you!
\end{frame}
}

\end{document}